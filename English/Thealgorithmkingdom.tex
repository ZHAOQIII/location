\documentclass{article}

\usepackage{ctex}

\usepackage{multicol}

\usepackage[top=1in, bottom=1in, left=1.25in, right=1.25in]{geometry}

\usepackage{lscape}

\usepackage{graphicx}

\usepackage{subfigure}

\author{Qi Zhao}

\date{April 25,2018}

\title{The algorithm kingdom}
\newcommand{\upcite}[1]{\textsuperscript{\textsuperscript{\cite{#1}}}}

\begin{document}


\maketitle
\par There is a possibility that China could be a close second to America-and perhaps even ahead of it in some areas of AI. And two things show it. First, Qi Lu, one of the bosses of Microsoft, would become chief operating officer at Baidu, Chinese leading search engine. Later the Association for the Advancement of Artificial Intelligence postponed its annual meeting. The planned date for the event in January conflicted with the Chinese New Year. Other evidence supports the claim that China had overtaken America in the number of published journal articles on the deep learning\upcite{Lecun2015Deep}, a branch of AI. Meanwhile, the number of AI-related patent submissions by Chinese researchers has increased by nearly 200 percent in recent years, although America is still ahead in absolute numbers (see chart).
\begin{figure}[htbp]
\centering
\includegraphics[width=0.5\textwidth]{11.eps}
\caption{Growth rate}
\label{1}
\end{figure}

\par When consider the inputs needed for AI, China is so well placed. Of the two most basic, computing power and capital, it has an abundance. And Chinese sheer size and diversity provide powerful fuel for this cycle. Above all another important source of support for AI in China is the government. So, in the race for the preeminence in AI, it will run America.
\section*{}
\bibliographystyle{elsarticle-num}
\bibliography{2}

\end{document}
