\documentclass[10pt,twocolumn,letterpaper]{article}

\usepackage{cvpr}
\usepackage{times}
\usepackage{epsfig}
\usepackage{url}
\usepackage{tipa}
\usepackage{float}
\usepackage{graphicx}
\usepackage{subfigure}
\usepackage{amsmath}
\usepackage{amssymb}
\usepackage[pagebackref=true,colorlinks,linkcolor=red,citecolor=green,breaklinks=true,bookmarks=false]{hyperref}
\cvprfinalcopy
\def\cvprPaperID{****}
\def\httilde{\mbox{\tt\raisebox{-.5ex}{\symbol{126}}}}
\setcounter{page}{1}
\begin{document}
\title{Skin Detection~\cite{Albiol2001Optimum}}
\author{Qi Zhao\\\\July 28, 2018}

\maketitle
\section{Introduction}
The goal of skin-detection algorithms is to infer a label w $\in$ \{0,1\} denoting the presence or absence of skin at a pixel, based on the RGB~\cite{Bo2013Unsupervised} measurements x = [$x^R$, $x^G$, $x^B$] at that pixel. This is a useful precursor to segmenting a face or hand, or it may be used as the basis of a crude method for detecting prurient content in web images. Taking a generative approach, it can be described the likelihoods as in Equation~\ref{equ:one}.
\begin{equation}\label{equ:one}
Pr(x|w = k) = Norm_x[\mu_k, \sigma_k]
\end{equation}

\par and the prior probability~\cite{Jeffreys1946An} over states as in Equation~\ref{equ:two}.

\begin{equation}\label{equ:two}
Pr(w) = Bern_w[\lambda]
\end{equation}

\par In the learning algorithm, it estimates the parameters $\mu_0$, $\mu_1$, $\Sigma_0$, $\Sigma_1$ from training data pairs \{$w_i$, $x_i$ \}$^I_{i=1}$ where the pixels have been labeled by hand. In particular we learn $\mu_0$ and $\Sigma_0$ from the subset of the training data where $w_i$ = 0 and $\mu_1$ and $\Sigma_1$ from the subset where $w_i$ = 1. The prior parameter is learned from the world states \{$w_i$\}$^I_{i=1}$ alone.
\par To classify a new data point x as skin or non-skin it can apply Bayes�� rule~\cite{Grether1980Bayes} in Equation~\ref{equ:three} and denote this pixel as skin if Pr(w = 1$|$x) $>$ 0.5. Figure~\ref{fig:onecol} shows the result of applying this model at each pixel independently in the image. Note that the classification is not perfect: there is genuinely an overlap between the skin and non-skin distributions and this inevitably results in misclassified pixels. The results could be improved by exploiting the fact that skin areas tend to be contiguous regions without small holes.
\begin{equation}\label{equ:three}
Pr(w = 1|x) = \frac{Pr(x|w = 1)Pr(w = 1)}{\sum_{k = 0}^1Pr(x|w = k)Pr(w = k)}
\end{equation}

\begin{figure}[htbp]
\centering
\includegraphics[width=0.9\linewidth]{skin.JPG}
 \caption{Skin detection. For each pixel we aim to infer a label w $\in$ \{0,1\} denoting the absence or presence of skin based on the RGB triple x. Here we modeled the class conditional density functions Pr(x$|$w) as normal distributions. a) Original image. b) Log likelihood (log of data assessed under class-conditional density function) for non-skin. c) Log likelihood for skin. d) Posterior probability of belonging to skin class. e) Thresholded posterior probability Pr(w$|$x) $>$ 0.5 gives estimate of w.}
\label{fig:onecol}
\end{figure}
\par The RGB data are naturally discrete with $x^R$, $x^G$, $x^B$ $\in$ \{0,1,...,255\}, which is basic of the skin detection model on this assumption. For example, modeling the three color channels independently, the likelihoods become in Equation~\ref{equ:four}.
\begin{equation}\label{equ:four}
Pr(x|w = k) = Cat_{x^R}[\lambda_k^R]Cat_{x^G}[\lambda_k^G]Cat_{x^B}[\lambda_k^B]
\end{equation}



\section{Conclusions}
\par Assume that the elements of the data vector are independent as na\"ive Bayes. Of course, it is not necessarily valid in the real world. To model the joint distribution~\cite{Longuet1975On} of the R,G, and B components, it might combine them to form one variable with $256^3$ entries and model this with a single categorical distribution. And it is more practical to quantize each channel to fewer levels (say 8) before combining them together.


{\small
\bibliographystyle{ieee}
\bibliography{50}
}


\end{document}

