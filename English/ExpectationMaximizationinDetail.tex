\documentclass[10pt,twocolumn,letterpaper]{article}

\usepackage{cvpr}
\usepackage{times}
\usepackage{epsfig}
\usepackage{url}
\usepackage{float}
\usepackage{graphicx}
\usepackage{subfigure}
\usepackage{amsmath}
\usepackage{amssymb}
\usepackage[pagebackref=true,colorlinks,linkcolor=red,citecolor=green,breaklinks=true,bookmarks=false]{hyperref}
\cvprfinalcopy
\def\cvprPaperID{****}
\def\httilde{\mbox{\tt\raisebox{-.5ex}{\symbol{126}}}}
\setcounter{page}{1}
\begin{document}
\title{Expectation Maximization~\cite{Bailey1995Unsupervised} in Detail}
\author{Qi Zhao\\\\August 29, 2018}

\maketitle
\section{Introduction}
The EM algorithm is used to find maximum likelihood~\cite{Dempster1977Maximum} or MAP estimates of model parameters $\theta$ where the likelihood Pr(x$|$$\theta$) of the data x can be written as in Equation~\ref{equ:one}.

\begin{equation}
\begin{split}
Pr(x|\theta) = \sum_kPr(x, h = k|\theta) = \sum_kPr(x|h = k, \theta)Pr(h = k)\\
Pr(x|\theta) = \int Pr(x, h|\theta)dh = \int Pr(x|h, \theta)Pr(h)dh
\end{split}
\label{equ:one}
\end{equation}

Where for discrete and continuous hidden variables, respectively. In other words, the likelihood Pr(x$|$$\theta$) is a marginalization of a joint distribution~\cite{Long2014Transfer} over the data and the hidden variables.



\section{Descriptions}

The EM algorithm relies on the idea of a lower bounding function (or lower bound), B[$\theta$] on the log likelihood. This is a function of the parameters $\theta$ that is always guaranteed to be equal to or lower than the log likelihood~\cite{Abou2018Unsupervised}. The lower bound is carefully chosen so that it is easy to maximize with respect to the parameters. This lower bound is also parameterized by a set of probability distributions \{$q_i$ ($h_i$)\}$^I_{i=1}$ over the hidden variables, so we write it as B[\{$q_i$ ($h_i$)\}, $\theta$]. Different probability distributions $q_i$ ($h_i$) predict different lower bounds B[\{$q_i$ ($h_i$)\}, $\theta$] and hence different functions of $\theta$ that lie everywhere below the true log likelihood (Figure~\ref{fig:one}).


\begin{figure}[H]
\centering
\includegraphics[width=0.5\textwidth]{em829.JPG}
 \caption{  Jensen��s inequality for the logarithmic function (discrete case). a) Taking a weighted average of examples E[y] and passing them through the log function. b) Passing the samples through the log function and taking a weighted average E[log[y]]. The latter case always produces a smaller value than the former (E[log[y]] $\leq$ log(E[y])): higher valued examples are relatively compressed by the concave log function.}
\label{fig:one}
\end{figure}

By iterating steps, the (local) maximum of the actual log likelihood is approached (Figure~\ref{fig:two}). To complete our picture of the EM algorithm, it must:

1.define B[\{$q_i$ ($h_i$)\}, $\theta^{[t?1]}$] and show that it always lies below the log likelihood;

2.show which probability distribution $q_i$ ($h_i$) optimizes the bound in the E-step;

3.show how to optimize the bound with respect to $\theta$ in the M-step;
\begin{figure}[hb]
\centering
\includegraphics[width=0.5\textwidth]{em8291.JPG}
 \caption{Expectation maximization algorithm. We iterate the expectation and maximization steps by alternately changing the distributions $q_i$ ($h_i$) and the parameter $\theta$ so that the bound increases. In the E-step, the bound is maximized with respect to $q_i$ ($h_i$) for fixed parameters $\theta$: the new function with respect to �� touches the true log likelihood at $\theta$. In the M-step, we find the maximum of this function. In this way we are guaranteed to reach a local maximum in the likelihood function.}
\label{fig:two}
\end{figure}





{\small
\bibliographystyle{ieee}
\bibliography{65}
}


\end{document}

