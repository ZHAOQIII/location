\documentclass[10pt,twocolumn,letterpaper]{article}

\usepackage{cvpr}
\usepackage{times}
\usepackage{epsfig}
\usepackage{url}
\usepackage{float}
\usepackage{graphicx}
\usepackage{subfigure}
\usepackage{amsmath}
\usepackage{amssymb}
\usepackage[pagebackref=true,colorlinks,linkcolor=red,citecolor=green,breaklinks=true,bookmarks=false]{hyperref}
\cvprfinalcopy
\def\cvprPaperID{****}
\def\httilde{\mbox{\tt\raisebox{-.5ex}{\symbol{126}}}}
\setcounter{page}{1}
\begin{document}
\title{Model Contingency~\cite{Fiedler1964A} of World on Data (Discriminative)}
\author{Qi Zhao\\\\July 22, 2018}

\maketitle
\section{Introduction}
At an abstract level, the goal of computer vision problems is to use the observed image data to infer something about the world. For example, it might observe adjacent frames~\cite{Lin1996Identification} of a video sequence and infer the camera motion, or it might observe a facial image and infer the identity.  There are two distinct approaches to modeling the relationship between the world state w and the data x, corresponding to modeling the posterior Pr(w$|$x)~\cite{Hinchey1996A}, or the likelihood Pr(x$|$w)~\cite{Dempster1977Maximum}. The two model types result in different approaches to inference. For the discriminative model~\cite{jaakkola1999exploiting}, it describe the posterior Pr(w$|$x) directly.
\section{Descriptions}
It defines a probability distribution over the world state w and make its parameters contingent on the data x. Since the world state is univariate and continuous, it chose the univariate normal. It fixes the variance, $\sigma^2$ and make the mean $\mu$ a linear function $\phi_0$ + $\phi_1$x of the data. So it can be seen in Equation~\ref{equ:one}.
\begin{equation}\label{equ:one}
Pr(w|x, \theta) = Norm_x[\phi_0 + \phi_1x, \sigma^2]
\end{equation}
\par where $\theta$ = {$\phi_0$, $\phi_1$, $\sigma^2$ } are the unknown parameters of the model (Figure~\ref{fig:onecol}). This model is referred to as linear regression.
\begin{figure}[H]
\centering
\includegraphics[width=0.5\textwidth,natwidth=530,natheight=188]{dis.JPG}
 \caption{Regression by modeling the posterior Pr(w$|$x) (discriminative). a) We model the world state w as normally distributed. b) We make the normal parameters a function of the observations x: the mean is a linear function $\mu$ = $\phi_0$ + $\phi_1$x of the observations, and the variance �� 2 is fixed. The learning algorithm fits the parameters $\theta$ = {$\phi_0$, $\phi_1$, $\sigma^2$ } to example training pairs {$x_i$, $w_i$}$^I_{i=1}$ (blue dots). In inference we take a new observation x and compute the posterior distribution Pr(w$|$x) over the state.}
\label{fig:onecol}
\end{figure}
The learning algorithm estimates the model parameters $\theta$ from paired training examples {$x_i$, $w_i$}$^I_{i=1}$. For example, in the MAP approach~\cite{shen2007map}, it need seek in Equation~\ref{equ:two}.
\begin{equation}
\begin{split}
\hat{\theta}& = argmax_{\theta}[Pr(\theta|w_{1...I}, x_{1...I})]\\
& = argmax_{\theta}[Pr(w_{1...I}|x_{1...I}, \theta)Pr(\theta)]\\
& = argmax_{\theta}[\prod_{i = 1}^{I} Pr(w_i|x_i, \theta)Pr(\theta)]
   \label{equ:two}
\end{split}
\end{equation}

\par where assumed that the I training pairs {$x_i$, $w_i$}$^I_{i=1}$ are independent, and defined a suitable prior Pr($\theta$).


{\small
\bibliographystyle{ieee}
\bibliography{47}
}


\end{document}

