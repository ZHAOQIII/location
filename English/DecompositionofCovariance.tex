\documentclass[10pt,twocolumn,letterpaper]{article}

\usepackage{cvpr}
\usepackage{times}
\usepackage{epsfig}
\usepackage{url}
\usepackage{float}
\usepackage{graphicx}
\usepackage{subfigure}
\usepackage{amsmath}
\usepackage{amssymb}
\usepackage[pagebackref=true,colorlinks,linkcolor=red,citecolor=green,breaklinks=true,bookmarks=false]{hyperref}
\cvprfinalcopy
\def\cvprPaperID{****}
\def\httilde{\mbox{\tt\raisebox{-.5ex}{\symbol{126}}}}
\setcounter{page}{1}
\begin{document}
\title{Decomposition of Covariance~\cite{Koutsoyiannis1999Optimal}}
\author{Qi Zhao\\\\July 20, 2018}

\maketitle
\section{Introduction}
Given a normal distribution~\cite{Stein1981Estimation} with mean zero and a full covariance matrix~\cite{Guevara2009Estimating}, it can be known that the iso-contours take an ellipsoidal form with the major and minor axes at arbitrary orientations. Now consider viewing the distribution in a new coordinate frame where the axes are aligned with the axes of the normal (Figure~\ref{fig:onecol}): in this new frame of reference, the covariance matrix $\Sigma_{diag}��$ diag will be diagonal. Denote the data vector in the new coordinate system by x�� = [$x_1$��, $x_2$��$]^T$ where the frames of reference are related by x�� = Rx. It can write the probability distribution~\cite{Rannala1996Probability} over x�� as in Equation~\ref{equ:one}.
\begin{figure}[H]
\centering
\includegraphics[width=0.8\textwidth,natwidth=510,natheight=242]{d.JPG}
 \caption{Decomposition of full co-variance. For every bivariate normal distribution in variables $x_1$ and $x_2$ with full covariance matrix, there exists a coordinate system with variables $x_1$' and $x_2$' where the covariance is diagonal: the ellipsoidal iso-contours align with the coordinate axes $x_1$' and $x_2$' in this canonical coordinate frame. The two frames of reference are related by the rotation matrix \textbf{R} which maps ($x_1$', $x_2$') to ($x_1$, $x_2$ ). From this it follows (see text) that any covariance matrix $\Sigma$ can be broken down into the product \textbf{$R^T\Sigma_{diag}'R$} of a rotation matrix \textbf{R} and a diagonal covariance matrix \textbf{$\Sigma_{diag}'$}.}
\label{fig:onecol}
\end{figure}
\begin{equation}\label{equ:one}
Pr(x��) = \frac{1}{{2\pi}^{\frac{D}{2}}\sqrt{|\Sigma_{diag}'|}}exp[-0.5x^{'T}\Sigma_{diag}^{'-1}x']
\end{equation}


If we convert back to the original axes by substituting in x 0 = Rx to get in Equation~\ref{equ:two}.
\begin{equation}
\begin{split}
Pr(x��)& = \frac{1}{{2\pi}^{\frac{D}{2}}\sqrt{|\Sigma_{diag}'|}}exp[-0.5(Rx)^{T}\Sigma_{diag}^{'-1}Rx]\\
& = \frac{1}{{2\pi}^{\frac{D}{2}}\sqrt{|R^T\Sigma_{diag}'R|}}exp[-0.5(x)^{T}(R^T\Sigma_{diag}'R)^{-1}x]
   \label{equ:two}
\end{split}
\end{equation}
where we have used $|R^T\Sigma��R|$ = $|R^T|$.$|\Sigma��|$.$|R|$ = 1.$|\Sigma��|$.1 = $|\Sigma��|$. Equation~\ref{equ:three} is a multivariate normal with covariance.
\begin{equation}\label{equ:three}
\Sigma_{full} = R^T\Sigma_{diag}'R
\end{equation}


\section{Conclusions}
It can be concluded that full covariance matrices are expressible as a product of this form involving a rotation matrix~\cite{Horn1954Doubly} R and a diagonal covariance matrix $\Sigma��$, which is possible to retrieve these elements from an arbitrary valid covariance matrix $\Sigma_{full}$ by decomposing it in this way using the singular value decomposition.
\par The matrix \textbf{R} contains the principal directions of the ellipsoid in its columns. The values on the diagonal of $\Sigma_{diag}��$ encode the variance (and hence the width of the distribution) along each of these axes. Hence the results of the eigen-decomposition can be used to answer questions about which directions in space are most and least certain.

{\small
\bibliographystyle{ieee}
\bibliography{46}
}


\end{document}

