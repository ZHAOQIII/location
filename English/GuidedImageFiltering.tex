\documentclass[10pt,twocolumn,letterpaper]{article}

\usepackage{cvpr}
\usepackage{times}
\usepackage{epsfig}
\usepackage{url}
\usepackage{float}
\usepackage{graphicx}
\usepackage{subfigure}
\usepackage{amsmath}
\usepackage{amssymb}
\usepackage[pagebackref=true,colorlinks,linkcolor=red,citecolor=green,breaklinks=true,bookmarks=false]{hyperref}
\cvprfinalcopy
\def\cvprPaperID{****}
\def\httilde{\mbox{\tt\raisebox{-.5ex}{\symbol{126}}}}
\setcounter{page}{1}
\begin{document}
\title{Guided Image Filtering}
\author{Qi Zhao\\\\June 14, 2018}

\maketitle
\section{Introduction}
In this paper, it proposes a novel type of explicit image filter - \emph{guided filter}~\cite{Yusuke2012Multispectral}. Derived from a local linear model, the guided filter generates the filtering output by considering the content of a guidance image, which can be the input image itself or another different image. The guided filter can perform as an edge-preserving smoothing operator like the popular bilateral filter~\cite{Barash2002Fundamental}, but has better behavior near the edges. It also has a theoretical connection with the matting Laplacian matrix~\cite{Trinajstic1994The}, so is a more generic concept than a smoothing operator and can better utilize the structures in the guidance image. Moreover, the guided filter has a fast and non-approximate linear-time algorithm, whose computational complexity is independent of the filtering kernel size. We demonstrate that the guided filter is both effective and efficient in a great variety of computer vision and computer graphics applications including noise reduction, detail smoothing/enhancement, HDR compression~\cite{Sun2008DHTC}, image matting/feathering, haze removal, and joint upsampling.

\section{Description}
It first defines a general linear translation-variant filtering process, which involves a guidance image I, an input image p, and an output image q. Both I and p are given beforehand according to the application, and they can be identical. The filtering output at a pixel i is expressed as a weighted average (in Eq.~\ref{fig:onecol}):
\begin{align}
q_i = \sum_{j}^{}W_{ij}(I)p_j
\end{align}
Where i and j are pixel indexes. The filter kernel W$_{ij}$ is a function of the guidance image I and independent of p. This filter is linear with respect to p.
\par Next we apply the linear model to all local windows in the entire image. Some further computations show that j W$_{ij}$ (I) = 1. No extra effort is needed to normalize the weights. Figure~\ref{fig:onecol} shows two examples of the kernel shapes in real images. In the top row are the kernels near a step edge. Like the bilateral kernel, the guided filter��s kernel assigns nearly zero weights to the pixels on the opposite side of the edge. In the bottom row are the kernels in a patch with small scale textures. Both filters average almost all the nearby pixels together and appear as low-pass filters~\cite{Hsieh2003Compact}.
\begin{figure}[t]
\begin{center}
\includegraphics[width=0.8\linewidth]{guide.JPG}
\end{center}
 \caption{Filter kernels. Top: a step edge (guided filter: r = 7, $\zeta$ = $0.1^2$, bilateral filter: $\varepsilon$$_s$ = 7, $\varepsilon$$_r$ = 0.1). Bottom: a textured patch (guided filter: r = 8, $\zeta$ = $0.2^2$, bilateral filter: $\varepsilon$$_s$ = 8, $\varepsilon$$_r$ = 0.2). The kernels are centered at the pixels denote by the red dots.}
\label{fig:long}
\label{fig:onecol}
\end{figure}
\section{Conclusions}

In this paper, it has presented a novel filter which is widely applicable in computer vision and graphics. Different from the recent trend towards accelerating the bilateral filter, it defines a new type of filter that shares the nice property of edge-preserving smoothing but can be computed efficiently and exactly. The filter is more generic and can handle some applications beyond the concept of �� smoothing ��.

{\small
\bibliographystyle{ieee}
\bibliography{28}
}


\end{document}

