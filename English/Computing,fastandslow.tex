\documentclass{article}

\usepackage{ctex}

\usepackage{multicol}

\usepackage[top=1in, bottom=1in, left=1.25in, right=1.25in]{geometry}

\usepackage{lscape}

\usepackage{graphicx}

\usepackage{subfigure}

\author{Qi Zhao}

\date{April 27,2018}

\title{Computing, fast and slow}
\newcommand{\upcite}[1]{\textsuperscript{\textsuperscript{\cite{#1}}}}

\begin{document}


\maketitle
\par IBM is not about to go down, but life will be tough in the cloud. IBM��s revenue has declined, year-on-year, for ten straight quarters. Because the industry is going though another wrenching change, which is some ways more challenging: instead of owing computers, businesses are increasingly renting computing services in the cloud. Not just the technology is on the move, but everything around it.
\par As corporate IT opens up to the outside world, the way in which it is developed, sold and used is changing too. Power is flowing from hardware engineers to software developers, the best of whom are in great demand. Meanwhile, customers are no longer willing just to buy the last technology; they want to pay for specific results.
\begin{figure}[htbp]
\centering
\includegraphics[width=0.5\textwidth]{IBM.jpg}
\caption{IBM's Rometty dives into the cloud}
\label{1}
\end{figure}

\par All this means that longer-established hardware firms have to offer a working environment that attracts younger people. So, IBM, an information-technology giant long known for its buttoned-down culture and blue business suits, starts to build the new interactive experience lab, where teams of employees from IBM and its customers jointly think up new online services and apps. Such project, IBM hopes, will help it grow again. The pressure mounts for IBM.


\end{document}
