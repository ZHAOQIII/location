\documentclass{article}

\usepackage{ctex}

\usepackage{multicol}

\usepackage[top=1in, bottom=1in, left=1.25in, right=1.25in]{geometry}

\usepackage{lscape}

\author{Qi Zhao}

\date{April 23,2018}

\title{Selling rides, not cars}
\newcommand{\upcite}[1]{\textsuperscript{\textsuperscript{\cite{#1}}}}

\begin{document}


\maketitle
\par A fully self-driving car expected by everyone will initially be offered for sale not to private owners but to robotaxi-fleet operators. That's two reasons, first, LIDAR sensors\upcite{Han2012Enhanced} are still so expensive, deployed in production cars, they would cost more than the rest the vehicle put together. Second, getting AVs to work safely and reliably is much easier if their geographical range is limited to places that have been mapped in fine detail, such as city centers. Now, Waymo, Uber and Voyage are testing and operating driverless taxis.
\par It's likely to be many years before AVs are cheap enough for individuals to buy them, and capable enough to operate outside predefined, geofenced areas. Meanwhile, the roll-out of cheap robotaxi in urban areas might encourage more young urbanites, who are ready going off car ownership anyway, to abandon it altogether. It already seems clear that AVs will cause the car industry and its adjacent business to change shape dramatically over the next couple of decades. Maybe, it will have far-reaching cultural and social effects too, most obviously in cities.
\section*{}
\bibliographystyle{elsarticle-num}
\bibliography{1}

\end{document}
