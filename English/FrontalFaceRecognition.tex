\documentclass[10pt,twocolumn,letterpaper]{article}

\usepackage{cvpr}
\usepackage{times}
\usepackage{epsfig}
\usepackage{url}
\usepackage{float}
\usepackage{graphicx}
\usepackage{subfigure}
\usepackage{amsmath}
\usepackage{amssymb}
\usepackage[pagebackref=true,colorlinks,linkcolor=red,citecolor=green,breaklinks=true,bookmarks=false]{hyperref}
\cvprfinalcopy
\def\cvprPaperID{****}
\def\httilde{\mbox{\tt\raisebox{-.5ex}{\symbol{126}}}}
\setcounter{page}{1}
\begin{document}
\title{Frontal Face Recognition~\cite{Georghiades2001From}}
\author{Qi Zhao\\\\September 4, 2018}

\maketitle
The goal of face identification (Figure~\ref{fig:one}) is to assign a label w $\in$ \{1...M\} indicating which of M possible identities the face belongs to based on a data vector x. The model is learned from labeled training data \{$x_i$, $w_i$\}$^I_{i=1}$ where the identity is known. In a simple system, the data vector might consist of the concatenated grayscale values from the face image, which should be reasonably large (say 50 X 50 pixels) to ensure that the identity is well represented.
\begin{figure}[H]
\centering
\includegraphics[width=0.5\textwidth]{904.JPG}
 \caption{ Face recognition. Our goal is to take the RGB values of a facial image x and assign a label w $\in$ \{1...K\} corresponding to the identity. Since the data are high-dimensional , we model the class conditional density function Pr(x$|$w = k) for each individual in the database as a factor analyzer. To classify a new face, we apply Bayes�� rule~\cite{Sandroni2005Efficient} with suitable priors Pr(w) to compute the posterior distribution~\cite{Brodersen2010The} Pr(w$|$x). We choose the label $\hat{w}$ =argma$x_w$ [Pr(w = k$|$x)] that maximizes the posterior. This approach assumes that there are sufficient training examples to learn a factor analyzer for each class.}
\label{fig:one}
\end{figure}


Since the data are high dimensional, a reasonable approach is to model each class conditional density function with a factor analyzer in Equation~\ref{equ:one}.
\begin{equation}\label{equ:one}
Pr(x_i | w_i = k) = Norm_{x_i}[\mu_k, \Phi_k \Phi_k^T + \Sigma_k]
\end{equation}


where the parameters for the $k^{th}$ identity $\theta_k$ = $\mu_k$, $\Phi_k$, $\Sigma_k$ can be learned from the subset of data that belongs to that identity using the EM algorithm~\cite{Dempster2015Maximum}. It also assign priors P(w = k) according to the prevalence of each identity in the database.

To perform recognition, it computes the posterior distribution Pr(w*$|$x*) for the new data example x* using Bayes�� rule. And it assign the identity that maximizes this posterior distribution. This approach works well if there are sufficient examples of each gallery individual to learn a factor analyzer, and if the poses of all of the faces are similar. In the next example, it develops a method to change the pose of faces, so that we can cope with the case where the poses differ.





{\small
\bibliographystyle{ieee}
\bibliography{68}
}


\end{document}

