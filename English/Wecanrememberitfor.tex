\documentclass{article}

\usepackage{ctex}

\usepackage{multicol}

\usepackage[top=1in, bottom=1in, left=1.25in, right=1.25in]{geometry}

\usepackage{lscape}

\usepackage{graphicx}

\usepackage{subfigure}

\author{Qi Zhao}

\date{April 29,2018}

\title{We can remember it for}
\newcommand{\upcite}[1]{\textsuperscript{\textsuperscript{\cite{#1}}}}

\begin{document}


\maketitle
\par Since 1946, ENIAC\upcite{Mccartney2001Eniac}, the first computer that could be operated by a single person, began flashing its ring counters, human beings and calculating machines have been on a steady march towards tighter integration. And the range of applications is wide, such as pockets, wrists and so on. The logical conclusion of all this is that computers will, one day, enter the brain.
\begin{figure}[htbp]
\centering
\includegraphics[width=0.5\textwidth]{ENIAC.jpg}
\caption{Elon Musk enters the world of brain-computer interfaces}
\label{1}
\end{figure}

\par ��Neural lace\upcite{Sanford2018Will}��, a science-fiction concept invented by Iain M. Banks, a novelist, that is, in essence, a machine interface woven into the brain. Although devices that can read and write data to and from the brain as easily as they would to and from a computer remain firmly in the realm if imagination, that has not stopped neuroscientists from indulging in some speculation.
\par Someone believes that human beings need to embrace implants to stay relevant in a world, which will soon be dominated by artificial intelligence. Proposing the artificial augmentation of human intelligence as a response to a boom in artificial intelligence may seem a bit much. Others hope to build devices for the treatment of neurological condition. Ultimately, they want to create cognition-enhancing implants that anyone might care to buy. The first brain implants were prosthetic visual system. And one of the latest ideas in the field is to read and interpret brain activity, in order to restore movement to the limbs of the paralysed. Though aimed initially at medical application, they also explicitly nod to the possible non-medical uses of this kind of implant technology.

\bibliographystyle{ieeetr}
\bibliography{3}
\end{document}
