\documentclass{article}

\usepackage{ctex}

\usepackage{multicol}

\usepackage[top=1in, bottom=1in, left=1.25in, right=1.25in]{geometry}

\usepackage{lscape}

\usepackage{graphicx}

\usepackage[colorlinks=true]{hyperref}

\usepackage{subfigure}

\author{Qi Zhao}

\date{May 5,2018}

\title{Question time}
\newcommand{\upcite}[1]{\textsuperscript{\textsuperscript{\cite{#1}}}}

\begin{document}


\maketitle
\par There is a phenomenon that women ask fewer questions than men\cite{Kuo2017Women} at seminars (see Fig. {\color{red}1}), for which the hypothesis is supported by data in a new working paper. One theory to explain the low share of women\cite{Holst2014Financial} in senior academic jobs is that they have less self-confidence than men.
\begin{figure}[htbp]
\centering
\includegraphics[width=0.35\textwidth]{question.jpg}
\caption{percentage of question and attendees}
\label{1}
\end{figure}
\par On average, half of each seminar��s audience was female. Men, however, were over 2.5 times more likely to pose question to the speakers-an action that may be viewed as a sign of greater competence. When a women did so, the gender split in question-asking was, on average , proportional to that of the audience. But if a man did so, there will be a reverse result depending on who asked the first question.
\bibliographystyle{ieeetr}
\bibliography{5}

\end{document}
